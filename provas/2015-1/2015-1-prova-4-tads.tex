\documentclass[12pt,a4paper]{article}
\usepackage{cmap} % Makes the PDF copiable. See http://tex.stackexchange.com/a/64198/25761
\usepackage[T1]{fontenc}
\usepackage[brazil]{babel}
\usepackage[utf8]{inputenc}
\usepackage{amsmath}
\usepackage{amsfonts}
\usepackage{amssymb}
\usepackage{amsthm}
\usepackage{textcomp} % \degree
\usepackage{gensymb} % \degree
\usepackage[usenames,svgnames,dvipsnames]{xcolor}
\usepackage{hyperref}
\usepackage{graphicx}
\usepackage[margin=2cm]{geometry}

\hypersetup{
    colorlinks = true,
    allcolors = {blue}
}

\newcommand{\fixme}{{\color{red}(...)}}

\newcommand*\tipo{PROVA IV}
\newcommand*\turma{TURMA B}
\newcommand*\disciplina{CDI0001}
\newcommand*\eu{Helder G. G. de Lima}
\newcommand*\data{26/06/2015}

\newcommand*\diff{\mathop{}\!\mathrm{d}}

\author{\eu}
\title{\tipo - \disciplina}
\date{\data}

\begin{document}
\thispagestyle{empty}
\newgeometry{margin=2cm,bottom=0.5cm}
\begin{center}
\includegraphics{udesc_joinville_cabecalho.pdf}
\\ Prof. \eu\footnote{
Este é um material de acesso livre distribuído sob os termos da licença \href{https://creativecommons.org/licenses/by-sa/4.0/deed.pt_BR}{Creative Commons BY-SA 4.0}.}

\noindent\begin{tabular}{l c c r}
  \textbf{\disciplina}
& \textbf{\tipo}
& \textbf{\data}
& \textbf{\turma}
\end{tabular}\vspace{-0.3cm}
\noindent\rule{17cm}{0.01cm}
\end{center}

\noindent Nome do aluno: \rule{14cm}{0.01cm}

%\section*{Instruções}
{\footnotesize
\begin{enumerate}
\renewcommand{\theenumi}{\Roman{enumi}}
\item Identifique-se em todas as folhas.
\item Mantenha o celular e os demais equipamentos eletrônicos desligados durante a prova.
\item Justifique cada resposta com cálculos ou argumentos baseados na teoria estudada.
\end{enumerate}
}
\noindent\rule{17cm}{0.01cm}
%\section*{Questões}
\begin{enumerate}
\item Escolha e resolva 5 das 7 integrais abaixo (cada uma valerá 2,0 pontos):
\begin{enumerate}
\item $\displaystyle\int \frac{2x^3-x-1}{2x^2-2x-4} \diff x$
\item $\displaystyle\int \operatorname{sen}^8(x) \cos^3(x) \diff x$
\item $\displaystyle\int \dfrac{1}{(4x+9)^4} \diff x$
\item $\displaystyle\int e^{4x} \cos(x) \diff x$
\item $\displaystyle\int x \sqrt{9x^2 + 1} \diff x$
\item $\displaystyle\int \frac{ \sqrt{16-x^2} }{x^2} \diff x$
\item $\displaystyle\int \sec^3(x) \diff x$
\end{enumerate}
\end{enumerate}

%\newpage
%\restoregeometry
%\section*{Respostas e observações}
%
%\begin{enumerate}
%\item \fixme
%\end{enumerate}

\end{document}
